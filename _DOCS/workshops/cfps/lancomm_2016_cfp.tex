\documentclass{sig-alternate-2013}

\usepackage[english]{babel}
\usepackage[latin1]{inputenc}
\usepackage{bbm}
\usepackage{url}
\usepackage{amsmath,pifont,amssymb,amsfonts}
\usepackage{color}
\usepackage{framed}
\usepackage{gensymb}
\usepackage{textcomp}
\usepackage{longtable}
\usepackage{array}
\usepackage{psfrag}
\usepackage{algorithm}
\usepackage{algorithmic}
\usepackage{cite}
\usepackage{multirow}

\usepackage{subcaption}

\makeatletter
\def\@copyrightspace{\relax}
\makeatother
\begin{document}

\title{\huge{LANCOMM 2016}\\~\\\vspace{-5mm}\Large{ACM SIGCOMM Workshop on 
Fostering Latin-American Research in Data Communication Networks}}

\maketitle

\section{Call for Papers}
\vspace{2mm}

The ACM SIGCOMM Latin American Workshop on Data Communication Networks -- 
LANCOMM -- aims to foster a higher representation of Latin American researchers 
working in data communication networks at SIGCOMM, as well as bridging Latin 
American research groups. LANCOMM serves as a meeting point for Latin American 
researchers to share new ideas and experiences and to discuss the challenges 
linked to the development of data communication networks in Latin America.

We solicit stimulating, original, previously unpublished ideas on completed 
work, position papers, and/or work-in-progress papers in the form of extended 
abstracts. We further encourage papers that propose new research directions or 
could generate lively debate at the workshop. We invite submissions on a wide 
range of networking research topics, including, but not limited to:

\begin{itemize}
\item Network architectures and algorithms
\item Experimental results from operational networks or network applications
\item Energy-aware communication
\item Network management
\item Network security and privacy
\item Network, transport, and application-layer protocols
\item Fault-tolerance, reliability, and troubleshooting
\item P2P, overlay, and content distribution networks
\item Resource management, QoS, and signaling
\item Routing, traffic engineering, switching, and addressing
\item Wireless, mobile, and sensor networks
\item Delay tolerant networks, satellite networks
\item Big data analytics and platforms in networking
\item Data mining, statistical modeling, and machine learning in networking
\item Innovative uses of network data beyond communication, IoT, and power grids
\item SDN, NFV and network programming
\end{itemize}

To foster participation of Latin American researchers at LANCOMM, we will give 
preference to papers coming from Latin American research groups, or addressing 
problems related to data communication networks of particular interest in Latin 
America.

\subsection{Submission Instructions}
\vspace{2mm}

Submitted extended abstracts must be in the form of a single PDF file of three 
(3) pages long (two-column 10pt ACM format), including figures, tables, and 
references. Papers must include author names and affiliations for single-blind 
peer reviewing by the PC. Authors of accepted papers are expected to present 
their papers at the workshop. Submissions must be original, unpublished work. 
Accepted papers will be published in the ACM Digital Library. Publication at 
LANCOMM does not preclude later publication.

\section{Important Dates}
\vspace{2mm}

\begin{itemize}
\item Paper registration deadline: 18th March 2016\vspace{-1mm}
\item Paper submission deadline: 25th March 2016\vspace{-1mm}
\item Paper acceptance notifications: 29th April 2016\vspace{-1mm}
\item Camera ready due: 20th May 2016\vspace{-1mm}
\end{itemize}

\section{Committees}
\vspace{2mm}
\noindent \textbf{Workshop Co-chairs}\vspace{-1mm}

\begin{itemize}
\item Rosa M.~M.~Le�o, Universidade Federal do Rio de Janeiro, 
Brazil\vspace{-1mm}
\item Fernando Paganini, Universidad ORT Uruguay, Uruguay\vspace{-1mm}
\item Javier Bustos-Jim�nez, Universidad de Chile \& NICLabs, 
Chile\vspace{-1mm} 
\item J.~Ignacio Alvarez-Hamelin, Universidad de Buenos Aires, 
Argentina\vspace{-1mm}
\item Pedro Casas, Austrian Institute of Technology \& Grupo ARTES, Austria \& 
Uruguay\vspace{-1mm}
\end{itemize}

\noindent \textbf{Workshop Advisor}\vspace{-1mm}

\begin{itemize}
\item Renata Cruz Teixeira, INRIA, France\vspace{-1mm}
\end{itemize}

\noindent \textbf{Program Committee}\vspace{-1mm}

\begin{itemize}
\item Pablo Belzarena, Universidad de la Rep�blica, Uruguay\vspace{-1mm}
\item Fabi�n Bustamante, Northwestern University, US\vspace{-1mm}
\item Sandra C�spedes, Universidad de Chile, Chile\vspace{-1mm}
\item Alessandro D'Alconzo, Austrian Institute of Technology, Austria\vspace{-1mm}
\item Diego Dujovne, Universidad Diego Portales, Chile\vspace{-1mm}
\item David Choffnes, Northeastern University, US\vspace{-1mm}
\item Benoit Donnet, Universit� de Li�ge, Belgium\vspace{-1mm}
\item Andr�s Ferragut, Universidad ORT Uruguay, Uruguay\vspace{-1mm}
\item Pablo Fierens, Insituto Tecnol�gico de Buenos Aires, 
Argentina\vspace{-1mm}
\item Kensuke Fukuda, National Institute of Informatics, Japan\vspace{-1mm}
\item Cecilia Galarza, Universidad de Buenos Aires, Argentina\vspace{-1mm}
\item Eduardo Gramp�n, Universidad de la Rep�blica, Uruguay\vspace{-1mm}
\item Diego Grosz, Instituto Balseiro, Argentina\vspace{-1mm}
\item Alejandro Hevia, Universidad de Chile, Chile\vspace{-1mm}
\item Nicol�s Hidalgo, Universidad de Santiago, Chile\vspace{-1mm}
\item Federico La Roca, Universidad de la Rep�blica, Uruguay\vspace{-1mm}
\item Marco Mellia, Politecnico di Torino, Italy\vspace{-1mm}
\item Daniel Menasch�, Federal University of Rio de Janeiro, Brazil\vspace{-1mm}
\item Antonio Rocha, Federal Fluminense University of Rio de Janeiro, 
Brazil\vspace{-1mm}
\item Erika Rosas, Universidad de Santiago, Chile\vspace{-1mm}
\item Ana Paula Silva, Federal University of Minas Gerais, Brazil\vspace{-1mm}
\end{itemize}
\end{document}