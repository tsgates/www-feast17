\documentclass[sigconf]{acmart}

\usepackage{booktabs} % For formal tables

\fancyhead{}
\settopmatter{printacmref=false, printfolios=false}



% Copyright
%\setcopyright{none}
%\setcopyright{acmcopyright}
%\setcopyright{acmlicensed}
%\setcopyright{rightsretained}
%\setcopyright{usgov}
%\setcopyright{usgovmixed}
%\setcopyright{cagov}
%\setcopyright{cagovmixed}


%% % DOI
%% \acmDOI{10.475/123_4}

%% % ISBN
%% \acmISBN{123-4567-24-567/08/06}

%% %Conference
%% \acmConference[WOODSTOCK'97]{ACM Woodstock conference}{July 1997}{El
%%   Paso, Texas USA} 
%% \acmYear{1997}
%% \copyrightyear{2016}

%% \acmArticle{4}
%% \acmPrice{15.00}

% These commands are optional
%\acmBooktitle{Transactions of the ACM Woodstock conference}
%% \editor{Jennifer B. Sartor}
%% \editor{Theo D'Hondt}
%% \editor{Wolfgang De Meuter}


%% \copyrightyear{2017} 
%% \acmYear{2017} 
%% \setcopyright{acmcopyright}
%% \acmConference{CCS'17}{}{October 30--November 3, 2017, Dallas, TX, USA}
%% \acmPrice{15.00}
%% \acmDOI{http://dx.doi.org/10.1145/XXXXXX.XXXXXX}
%% \acmISBN{ISBN 978-1-4503-4946-8/17/10} 
%% %Authors, replace the red X's with your assigned DOI string. See pdf attached to ACM rightsreview confirmation email.

\fancyhead{}
\settopmatter{printacmref=false, printfolios=false}



\begin{document}
\title{FEAST 2017: The Second Workshop on Forming an Ecosystem Around Software Transformation}
%% \titlenote{Produces the permission block, and copyright information}
%% \subtitle{Extended Abstract}
%% \subtitlenote{The full version of the author's guide is available as
%%   \texttt{acmart.pdf} document}


\author{Taesoo Kim}
%% \authornote{Dr.~Trovato insisted his name be first.}
%% \orcid{1234-5678-9012}
\affiliation{%
  \institution{Georgia Institute of Technology}
  %% \streetaddress{P.O. Box 1212}
  \city{Atlanta} 
  \state{GA 30332, USA} 
  \postcode{30332}
}
\email{taesoo@gatech.edu}

\author{Dinghao Wu}
%\authornote{The secretary disavows any knowledge of this author's actions.}
\affiliation{%
  \institution{The Pennsylvania State University}
  %\streetaddress{P.O. Box 1212}
  \city{University Park} 
  \state{PA 16802, USA} 
  \postcode{16802}
}
\email{dwu@ist.psu.edu}


% The default list of authors is too long for headers}
\renewcommand{\shortauthors}{Taesoo Kim and Dinghao Wu}


\begin{abstract}
The Second Workshop on Forming an Ecosystem
Around Software Transformation (FEAST 2017)
is held in conjunction with the 24th ACM
Conference on Computer and Communications
Security (CCS 2017) on November 3, 2017 in Dallas, Texas. The
workshop is geared toward discussion and
understanding of several critical topics surrounding
software executable transformation for improving
the security and efficiency of all software used in
security-critical applications. The scope of
discussion for this workshop includes topics that
may be necessary to fully exploit the power and
impact of late-stage software customization effort.
\end{abstract}


\copyrightyear{2017}
\acmYear{2017}
%\setcopyright{acmcopyright}
\setcopyright{rightsretained}
\acmConference{CCS '17}{October 30-November 3, 2017}{Dallas, TX,
USA}\acmDOI{10.1145/3133956.3137052}
\acmISBN{978-1-4503-4946-8/17/10}

%
% The code below should be generated by the tool at
% http://dl.acm.org/ccs.cfm
% Please copy and paste the code instead of the example below. 
%
\begin{CCSXML}
<ccs2012>
<concept>
<concept_id>10002978.10003022.10003465</concept_id>
<concept_desc>Security and privacy~Software reverse engineering</concept_desc>
<concept_significance>500</concept_significance>
</concept>
<concept>
<concept_id>10002978.10003022.10003023</concept_id>
<concept_desc>Security and privacy~Software security engineering</concept_desc>
<concept_significance>300</concept_significance>
</concept>
<concept>
<concept_id>10011007.10011006.10011073</concept_id>
<concept_desc>Software and its engineering~Software maintenance tools</concept_desc>
<concept_significance>300</concept_significance>
</concept>
<concept>
<concept_id>10011007.10011074.10011099</concept_id>
<concept_desc>Software and its engineering~Software verification and validation</concept_desc>
<concept_significance>300</concept_significance>
</concept>
</ccs2012>
\end{CCSXML}

\ccsdesc[500]{Security and privacy~Software reverse engineering}
\ccsdesc[300]{Security and privacy~Software security engineering}
\ccsdesc[300]{Software and its engineering~Software maintenance tools}
\ccsdesc[300]{Software and its engineering~Software verification and validation}


\keywords{Software transformation, binary code, security, debloating}

\maketitle

\section{Introduction}
Typical software engineering methodologies are largely focused on
programmer productivity and their methods have been known to introduce
significant execution inefficiency as a side effect.  Recent work
investigating efficient and timely software has attempted to enhance
software execution efficiency while preserving the source code-level
abstractions and object-orientation that enhance a programmer's
productivity.

Such efforts seek to undo the side effects on security and performance
overhead by reclaiming software execution efficiency and reducing
indirection, as well as performing automatic program de-layering and
program specialization (de-bloating). Several promising results from
these efforts have demonstrated their viability in improving program
execution efficiency as well as reduction of the cyber security attack
surface. As a result, the community may benefit by investing in the
development of tool ecosystems to take advantage of this recent
progress, to mature the technologies, and determine how best to
transparently deploy them.

Despite some early progress within the research community, software
executable transformation is not a solved science. Some critical
problems reverse engineering and binary understanding are, in the
general case, undecidable. Various automated tools and ecosystems will
need to be investigated and developed to guarantee the effectiveness
and correctness of transformation efforts and to enhance and ensure
the security of transformed software. The FEAST workshop aims at
forming an ecosystem Around Software Transformation.

\section{Topics}
The FEAST workshop will provide a forum for researchers to share
ideas and development on software transformation.
It includes topics geared toward:
\begin{itemize}

\item Understanding issues of software executable transformation for
  various programming languages and environment, and the potential
  methods for alleviating those issues.

\item Identification of tools to be investigated and developed for
  guaranteeing correctness, enhancing security, and enabling
  non-critical/undesired feature removal.

\item Identification of layers and areas of computing systems that are
  suitable for and can benefit from software
  customization/transformation, along with identification of
  associated challenges and constraints, and the particular adaptation
  to the methodology needed to operate within the identified areas.

\item Automated extraction of models from software executables that
  are amenable to formal methods analysis and verification.

\end{itemize}

\section{Format}
Submissions should be in two-column, 10-point
format, and can be up to 6 pages in length with as
many additional pages as necessary for references.

\section{Review Process}
All accepted papers received two to three double-blind reviews.
We would like to thank all PC members and external reviewers for
their contributions.

\section{Invited Speakers}
The workshop has an invited talk. % s on XXX and YYY.
The keynote speaker is Vikram S. Adve from 
University of Illinois at Urbana-Champaign.
%% \begin{itemize}
%% \item X and
%% \item Y.
%% \end{itemize}

%\newpage
\pagebreak

\section{Workshop Organization}

\noindent
\textbf{Program Committee Chairs}
\begin{itemize}
\item Taesoo Kim, \textit{Georgia Tech}%, Co-chair
\item Dinghao Wu, \textit{Penn State University}%, Co-chair
\end{itemize}

\noindent
\textbf{Program Committee Members}
\begin{itemize}
\item Kevin Hamlen, \textit{UT Dallas}
%\item Taesoo Kim, \textit{Georgia Tech}, Co-chair
\item Wenkee Lee, \textit{Georgia Tech}
\item Zhiqiang Lin, \textit{UT Dallas}
\item Mayur Naik, \textit{UPenn}
\item Mathias Payer, \textit{Purdue University}
\item Aravind Prakash, \textit{Binghamton University}
\item Binoy Ravindran, \textit{Virginia Tech}
\vfill\eject 
\item Yan Shoshitaishvili, \textit{Arizona State University}
\item Gang Tan, \textit{Penn State University}
\item Jan Vitek, \textit{Northeastern University}
\item XiaoFeng Wang, \textit{Indiana University}
%\item Dinghao Wu, \textit{Penn State University}, Co-chair
\item Dongyan Xu, \textit{Purdue University}
\item Harry Xu, \textit{UC Irvine}
\item Daphne Yao, \textit{Virginia Tech}
\end{itemize}

%\section{Conclusions}

\vspace*{6pt}
\begin{acks}
The FEAST Workshop is sponsored by Association for Computing Machinery
(ACM), ACM SIGSAC, and Office of Naval Research (ONR).
\end{acks}

\bibliographystyle{ACM-Reference-Format}
%% \bibliography{sample-bibliography} 

\end{document}
